\newpage
\section{Introduction}

The ability to lock a laser to a very specific frequency is of great importance to the UBC Quantum Degenerate Gases Laboratory.  They use these optical systems to laser cool atomic and molecular gases.  This requires a very narrow laser linewidth, and a stable carrier frequency.  Any improvement in the frequency precision of the laser will increase the quantity of atoms that can be trapped or cooled, as well as the attainable temperatures in a trap.  Often, this frequency must be very close to the natural transitions of the gas. \\

One way to acquire such a system is to pass the beam through a gas.  The gas has state transitions at particular optical frequencies, which can be detected by sweeping the probing laser. The gas will absorb photons most strongly at the atomic transition resonances, which will appear as sharp dips in the transmission spectrum.  However, since the gas is at room temperature, the broad absorption dip will have a bandwidth of about 1 GHz, which is too wide for accurate laser locking.

The Madison group has an existing system to lock lasers to various atomic transitions using AOM-based saturated absorption spectroscopy and the Pound-Drever-Hall method. However, this system is limited in a number of ways: \\
\begin{tabularx}{\linewidth}{lX}
  Stability & Currently, the signal is processed in a way that leaves it sensitive to acoustic perturbations. As such, it has a large jitter amplitude during operation, which results in higher laser linewidths and, occasionally, total loss of locking. \\
  Bandwidth & Due to the low bandwidth of various components (including the modulating element) the signal's bandwidth is limited to approximately 10 kHz. This reduces the speed at which the laser servo can acquire a lock. \\
  Zero Crossings & Due to AM pickup, shifts of alignment and laser power, the null locking points actually shift around during operation. As stated, shifts away from precise atomic transitions reduce the efficiency of experiments.
\end{tabularx} \\ \ \\
This project aims to improve various elements of this feedback system to improve stability, bandwidth and SNR and minimize the need for manual tuning. Furthermore, the methods employed here remove DC shifts in the error signal zero crossings. These improvements will lead to a master laser that has smaller linewidth and a more accurate/stable carrier frequency.

\subsection{Doppler Effects}

The gas in a vapour cell is at room temperature. In a saturated absorption spectroscopy scheme, for example, some atoms will be moving towards the probe beam, and other atoms will be moving away from it. As a result, if a single laser is used, the atoms moving towards the laser will perceive a higher frequency, and thus will excite at lower frequencies than the resonance frequency.  The opposite is also true.  If the absorption-vs-frequency curve is measured for a laser passed through the gas, there is a wide smear that is approximately 1 GHz wide \cite{madison14}.  This makes the apparent spectrum of the gas much wider, and thus, much harder to lock a signal to. This is depicted in \textbf{Figure \ref{fig:doppler}}. \\

Redirecting another single frequency laser around the sample, to the opposite side, will yield large improvements in feature size.  Now, when the sweep laser is at the same frequency as the pumping laser, its light will not be absorbed by the atoms with zero horizontal velocity.  This creates a Doppler-cancelled feature, which is about 10--100MHz wide.

\begin{figure}[H]
    \centering
    \includegraphics[width=0.45\textwidth]{figures/doppler.png}
    \caption[Doppler effect demonstration]{Doppler effect demonstration.  The probe beam is absorbed across a wide frequency range, but only the non-Doppler-shifted particles are affected by both beams from left and right.}
    \label{fig:doppler}
\end{figure}

\subsection{Acousto-Optic Modulator}

One way to modulate the frequency of the beam is to use an acousto-optic modulator (AOM). An AOM is a device that acoustically varies its shape, adjusting the distance the light has to travel to reach its destination.  By driving the optical medium at a particular frequency, an incident beam of light is diffracted and Doppler shifted depending on what angle it is sampled at. Modulating this driving frequency at an additional frequency $\Omega$, produces a beam of light that is phase modulated at a frequency $\Omega$. \\

AOMs are limited by the mechanical properties of the optical medium. Specifically, while being driven at some nominal frequency that may reach up to 100s of MHz, they can only register a frequency modulation at a rate up to several MHz. This is due to the fact that the rate of acoustic propagation across the medium is limited. Depending on what refraction angle it is sampled at, the output beam of the acousto-optic modulator also contains a shift in the carrier frequency.  This shift needs to be accounted for in a later stage of the system.

\subsection{Electro-Optic Modulator}

An electro-optic modulator (EOM), is a device that uses an electric field to modulate the phase of an EM wave.  These devices exploit either the Kerr or Pockels effects, whereby the index of refraction in a crystal is affected by the presence of an electric field.  Therefore, by adjusting the electric field in the crystal, the phase (and ultimately frequency spectrum) of the beam can be varied. The EOM does \emph{not} shift the carrier frequency, unlike an AOM. Furthermore, the resultant phase modulation frequency of the incident laser can reach into the GHz range, orders of magnitude above the
capability of an AOM.\\

Unfortunately, for large modulation depths, EOMs require fairly large driving voltages, on the order of hundreds of volts.  Some manufacturers provide a full EOM solution, with built-in amplifier, to make their equipment easier to use, making an external high-voltage amplifier unnecessary. This project makes use of a home-built EOM/driver system based around a Lithium Niobiate (LiNbO$_3$) crystal.

\subsection{The Pound-Drever-Hall Method}

Optical frequencies generally exceed 100 THz. It is not possible to extract information at these frequencies using conventional electronic measurement equipment. Systems make use of schemes that down-mix these optical signals into RF range, which are then processed via conventional means (RF electronics and digital systems). \\

The Pound-Drever-Hall method makes use of phase-modulated light to measure an optical response and down-mix it into RF range.  This signal is then mixed again with the same oscillator to produce a DC signal. Changing the modulation frequency varies the resolution of this derivative function. At extremes, this derivative function is either small and featureless, or is excessively noisy and not useful. \\

This method is used to lock to extrema of absorption spectra.  In an atomic gas, these features are typically transition and crossover resonances \cite{maguire2006}. Carefully selecting a modulation frequency to obtain a useful error signal about a point of interest is a principal challenge of implementing this method.

\subsection{Spectroscopy Methods}

There are a variety of methods to generate an absorption spectrum from an atomic gas, involving different configurations of cell/beam alignment, beam power ratios and various signal processing techniques. The original proposal involved the use of saturated absorption spectroscopy, which is elaborated on in \textbf{Section \ref{sec:sat_abs}}. However, due to various parasitic effects and poor SNR, (see \textbf{Figure \ref{fig:sat_abs_bad}}) the final configuration made use of a modulation transfer scheme, which has been described well in the literature \cite{Shirley:82}. There have also been detailed experiments specifically involving rubidium, which were a good starting point for this project \cite{0957-0233-19-10-105601}. These processes are described in more detail in \textbf{Section \ref{sec:theory}}.

\begin{figure}
  \begin{tabular}{cc}
    \includegraphics[width=0.47\textwidth]{figures/{AOM_85}.png} &
    \includegraphics[width=0.47\textwidth]{figures/{AOM_87}.png} \\
  \end{tabular}
  \caption[Existing AOM-based, error signals]{Existing AOM-based, single demodulation channel error signals around Rb85 (left) and Rb87 (right) D2 transitions. Each figure shows the reference spectrum (black), and the resulting PDH error signal (red).}
  \label{fig:aom_spectra}
\end{figure}

\begin{figure}
    \centering
    \includegraphics[width=0.5\textwidth]{figures/best_sat_abs.png}
    \caption[Acquired saturated absorption spectroscopy error signals]{Saturated absorption spectroscopy based PDH signal acquired near the start of the project. Shown is the ramping signal (yellow), AOM-based reference saturated absorption spectrum (blue) and the new EOM-based PDH signal (pink). This signal is suboptimal due to low SNR (approximately 3:1), which was difficult to improve given avai,lable resources. Additionaly, AM pickup effects bring into question the accuracy of the zero crossings.}
    \label{fig:sat_abs_bad}
\end{figure}
