\newpage
\section{Conclusion}

The feedback system built here, based around an EOM and modulation transfer spectroscopy, provides the Madison Group researchers with a more robust method of precisely locking a laser to an atomic transition resonance of Rubidium. \\

Construction was completed on a optical table setup based around a Rubidium vapour cell, and a Pound-Drever-Hall type error signal was generated using a photosensor and discrete RF components. Optical modulation was achieved with an existing home-built EOM with a Lithium Niobiate crystal and corresponding driving electronics.\\

The new system offers a signal bandwidth of 10MHz, though this can be pushed closer to the EOM modulation frequency with better filter solutions. It was shown that the locking points are stable, and that their resultant frequency domain noise is comparable with the existing AOM system, assuming identical laser servo behaviour. Though the SNR in the data sets is lower than desired, it can easily be improved by a signifcant margin. Substituting suboptimal electrical components will increase SNR by a factor of four. Signal averaging can also improve this significantly, but at a cost of bandwidth. \\

\section*{Acknowledgements}
\addcontentsline{toc}{section}{Acknowledgements}

This project was aided immensely by the research staff in the UBC Quantum Degenerate Gases laboratory led by Dr. Kirk Madison. Dr. Madison provided direct guidence on theoretical matters and contributed significant physical resources. Gene Polovy and Will Gunton of the Madison Group were extremely helpful with the physical setup, and with locating equipment. Janelle van Dongen provided significant assistance with the laser control instrumentation and data acquisition. Further thanks goes to Dr. David Jones, Dr. Jim Booth, Mariusz Semczuk, Koko Yu, Kahan Dare and Kais Jooya for their suggestions and guidance.