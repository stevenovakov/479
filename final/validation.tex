\newpage
\section{Validation}
\label{sec:validation}

\subsection{Data Collection}

Characteristic error signals were acquired for the Rb85 D2 $\left|F=3\right\rangle \rightarrow F'$ and Rb87 D2 $\left|F=2\right\rangle \rightarrow F'$ spectra for a variety of power/temperature combinations. For each spectrum, variable pump/probe power combinations were measured at room temperature. This is shown in \textbf{Appendix \ref{app:85pwr}, \ref{app:87pwr}}. From these sets, a power combination with a high SNR and large locking point slope was chosen to measure the effect of heating. As data processing was not performed until later, the best choice at the time was thought to be 4.0 mW for both pump and probe power. At these power levels the temperature was varied up to $\sim 100^\circ$ C, shown in \textbf{Appendix \ref{app:85temp}, \ref{app:87temp}}. It is clear from the data that the optimal temperature, with respect to slope/noise tradeoff, is likely between $20-40^\circ$ C and that the pressure broadening rapidly deteriorates the signal past $60^\circ$ C. \\

The SNR of the higher quality error signals is approximately 5:1 (e.g. 4.0 mW, 6.0 mW probe series). Though this seems poor compared to the existing AOM solution ($\sim$10:1), it is thought to mostly be a result of suboptimal electrical components. Simple improvements that can increase SNR by a factor of four or higher are suggested in \textbf{Section \ref{sec:recommendations}}.

\subsection{Locking Features}

Finding the desired locking point features first involved establishing a MHz/s scale on each of the time-domain spectra collected by the oscilloscope. All data sets were normalized to the reference AOM saturated absorption spectrum which was constant for each transition throughout data acquisition. This background spectrum was passed through a Savitzky-Golay piecewise polynomial filter to remove the higher frequency noise. Then, the two most prominent peaks correspond to the $\left|F=2\right\rangle \rightarrow F'=2,3$ and  $\left|F=2\right\rangle \rightarrow F'=1,3$ crossover resonances in Rb87 and the $\left|F=3\right\rangle \rightarrow F'=3,4$ and  $\left|F=3\right\rangle \rightarrow F'=2,4$ crossover resonances in Rb85.  The values of these resonances are known to high precision \cite{steckrb85, steckrb87}. Using their location, one can convert from the time domain to the frequency domain. \\

There are two primary sources of error in establishing these MHz/s scales. First, the peak-finding filter will have some error associated with it. As this is a piecewise interpolant and not a regression fit, it is difficult to specify quantitative value for the fit error, except by visual inspection. The second source of error is the laser ramping itself. The ramping signal for the laser servo is a low noise, nearly ideal triangle wave. However, the actual frequency response of the laser is not linear with respect to this ramp. The actual frequency of the laser will oscillate about the ramp, as per typical control system responses, thereby shrinking and stretching the frequency scale all across the data sets. The stated error in the frequency domain is a simple quadrature error that uses the standard deviation of the fit as a parameter. The sponsor is in agreement that this is a sufficient metric. \\

Questions were raised about the locations of the resonances across the data sets, and whether they moved in an obvious pattern as the pump:probe power ratios were changed. The locations of the modulation transfer spectrum peaks were acquired by applying a Savitzky-Golay filter to the modulation transfer spectrum and locating the extrema. As can be seen in \textbf{Figure \ref{fig:peak_drift}}, their measured location changes with no perceivable pattern. Furthermore, all of the data points are within each other's error bars in the frequency domain, and can therefore be considered static. \\

The slope and noise about the optimal locking points in each spectrum were derived by a simple linear regression in a small window about the resonance peaks. The fitted line was subtracted from the data and the min/max of the resultant signal in that same window was used to determine the noise in mVpp. It may be prudent to instead compute RMS noise in the window, but, upon visual inspection, the background noise signal was consistent in amplitude across the spectrum, and so they are likely to be closely related by the $\sqrt{2}$ ratio of an ideal, monochromatic sinusoid. Given the data for the 6.0 mW:6.0 mW data sets, the resulting frequency domain noise about these points is expected to be $\sim$4.52 MHz-pp for Rb87 and $\sim$3.82 MHz-pp for Rb85. Comparively, the AOM data shown in \textbf{Figure \ref{fig:aom_spectra}} yields $\sim$0.88 MHz-pp for Rb87 and $\sim$0.46 MHz-pp for Rb85. It is not easy to determine what effective laser linewidth this would result in for a laser that is locking to the null point, rather than being swept. Additionally, the AOM data is filtered down to approximately 10 kHz, and while the SNR is high, it is difficult to tell how this bandwidth limitation would affect the laser servo. Additionally, making the proposed component improvements would bring the new system's frequency noise down by a factor of four, at minimum, putting it on-par with the AOM system, without the bandwidth limitations.

\begin{figure}
  \begin{tabular}{cc}
    \includegraphics[width=0.47\textwidth]{figures/{85_peak_drift}.png} &
    \includegraphics[width=0.47\textwidth]{figures/{87_peak_drift}.png} \\
  \end{tabular}
  \caption[Modulation transfer, locking point drift]{ Behaviour of modulation transfer peaks as a function of varying probe/pump power. Only the three most prominent peaks were consistently resolvable. The error is calculated from the scale fitting variance, and increases for higher frequency regions.}
  \label{fig:peak_drift}
\end{figure}
