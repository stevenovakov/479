\newpage

\section{Recommendations}
\label{sec:recommendations}

\subsection{Improved Phase Adjustment}

Current understanding of modulation transfer spectroscopy indicates that the position of the zero crossing is independant of the phase adjustment.  However, as shown in prior work, the amplitude of the signal is not \cite{0957-0233-19-10-105601}. The onboard phase adjustment is limiting in its range, as shown in \textbf{Figure \ref{fig:eom_phase}}.  Implementing an improved phase adjustment which allows a full $\pi/2$ would make this optimization much easier to do.

\subsection{EOM driving frequencies}

The features in the Rb85 transitions are roughly 30MHz apart.  With a 20MHz EOM driver signal, it was noticed that the fringes of two strong signals overlap, crossing zero at a crossover.  If a 30MHz EOM driver were used instead, these two signals might constructively amplify, creating a very clean error signal. Since the EOM driver oscillator frequency was non-adjustable, it was not possible to characterize its effect on the error signal.

\subsection{Laser Power}

The highest quality error signal obtained made used 100\% of available laser power.  It is possible that by increasing the power even further, an even-better signal-to-noise ratio could be achieved.

\subsection{Improved pre-amplifiers}

The most obvious path to noise improvement is selecting better amplifiers.  The amplifiers used here, MiniCircuits ZFL-1000+, also exists in a low noise variant, the ZFL-1000LN+.  This amplifier has nearly identical properties, but the noise figure is lower, 3dB instead of 6dB \footnote{Noise figure is the amount that the SNR is increased by the amplifier.  Given a -20dBm carrier, with -30dBm of noise, and a +20dB amplifier with +6dB noise figure, the output signal will contain a 0dBm carrier, and -14dBm of noise, changing the SNR from 10dB to 4dB.}. The best locking features in \textbf{Appendices \ref{app:85pwr} - \ref{app:87temp}} have an SNR of approximately 5:1. The existing AOM signal (\textbf{Figure \ref{fig:aom_spectra}}), has an SNR of approximately 10:1. Switching out the existing amplifiers for a better model should increase the SNR by 6dB, to a value of 20:1 (two amplifiers in series). This is more robust than the existing AOM solution.

Additionally, the amplifiers used do not have the desired specifications.  They are the most appropriate choice given the supplier (Mini-Circuits), but they are over-specified.  These amplifiers are designed to work up to 1000 MHz, whereas an ideal cutoff would be in the range of 25-30 MHz.  There are likely amplifiers better suited to this purpose.  If not, then one could probably be made which meets these specifications, and has very good performance.  See, for example TI's guide to RF op amp circuits.\cite{ti_amps}.

\subsection{Additional Low-Pass Filtering}

A spectral analysis of the noise indicates that the energy distribution is flat in frequency, white noise, up to the low-pass cutoff frequency (10MHz).  A crude, but likely effective way to remove noise would be to bring in a lower low-pass filter, which would have the effect of `averaging' the error signal in time, reducing the amount of white noise, at the cost of response time.


\subsection{Closed Loop Operation}

In order for the system to actually be used, the feedback loop would need to be closed.  That is, the demodulated error signal would need to be fed back into a laser diode controller, which would use the signal to adjust the frequency of the laser itself.  A suitable solution available in the lab was the Vescent D2-105 laser controller, which includes fast (current) and slow (piezo) control, and a built in PI$^2$D controller.  This was not implemented, as the sponsor indicated that they would prefer a deeper investigation of possible tactics to improve the error signal, versus closing the loop and providing a functional (but sub-optimal) solution.

