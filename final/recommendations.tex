\newpage

\section{Recommendations}

\subsection{Improved Phase Adjustment}

Our current understanding of modulation transfer spectroscopy indicates that the position of the zero crossing is independant of the phase adjustment.  However, the amplitude of the signal is not.  This parameter is fairly easy to optimize, but the current phase adjustment scheme does not work very well.  Implementing an improved phase adjustment which allows a full 180° would make this optimization much easier to do.

\subsection{EOM driving frequencies}

The features in the Rb85 transitions are roughly 30MHz apart.  With a 20MHz EOM driver signal, we notice that the fringes of two strong signals overlap, crossing zero at a crossover.  If a 30MHz EOM driver were used instead, these two signals might constructively amplify, creating a very clean error signal.

An alternative approach uses the Rb87 features, which are generally more separated from one another.  Since our oscillator was fixed in frequency, we didn't get to experiment with the effect of the driving frequency on the quality of the error signal.

\subsection{Laser Power}

The best quality measurement we made was using 100\% of our laser's power.  Thus, it is possible that by increasing the power even further, an even-better signal-to-noise ratio could be achieved.

\subsection{Improved pre-amplifiers}

Near the end of the project, we decided that reducing noise was our priority.  The first and foremost concern is the noise from the amplifiers themselves.  The amplifiers we used, the ZFL-1000+, also exists in a low noise variant, the ZFL-1000LN+.  This amplifier has nearly identical properties, but the noise figure is lower, 3dB instead of 6dB \footnote{Noise figure is the amount that the SNR is increased by the amplifier.  That is, if we have a -20dBm carrier, with -30dBm of noise, and we feed it into a +20dB amplifier with +6dB noise figure, we get out a signal with 0dBm carrier, and -14dBm of noise.  So our SNR goes from 10dB to 4dB.}

Additionally, the amplifiers we're using do not have the specifications we desire.  They are the most appropriate from the supplier we used (minicircuits), but they're still far over-specified.  These amplifiers are designed to work up to 1000MHz, whereas we only need operation up to 20MHz.  We also require only a very narrow bandwidth near 20MHz.  There are likely amplifiers better suited to this purpose.  If not, then one could probably be made which meets these specifications, and ideally, has very good performance.  See, for example TI's guide to RF op amp circuits\cite{ti_amps}.

\subsection{Additional Low-Pass Filtering}

A spectral analysis of our noise indicates that the energy distribution is flat in frequency, white noise, up to our low-pass cutoff frequency (10MHz).  A crude, but likely effective way to remove noise would be to bring in a lower low-pass filter, which would have the effect of "averaging" the error signal in time, reducing the amount of white noise, at the cost of response time.


\subsubsection{Close the Loop}

In order for the system to actually be used, the feedback loop would need to be closed.  That is, the demodulated error signal would need to be fed back into a laser diode controller, which would use the signal to adjust the frequency of the laser itself.  A suitable solution available in the lab was the Vescent D2-105 laser controller, which includes fast (current) and slow (piezo) control, and a built in PI²D controller.  This was not implemented, as the sponsor indicated that they would prefer a deeper investigation of possible tactics to improve the error signal, versus closing the loop and providing a functional (but sub-optimal) solution.



