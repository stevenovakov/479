\newpage
\section*{Executive Summary}

This project, sponsored by the UBC Quantum Degenerate Gases
Laboratory---Madison Group, aims to build and characterize a high
fidelity, small linewidth, laser frequency locking system based on the
Pound-Drever-Hall (PDH) locking method. \\

Existing systems at the sponsors' laboratory are already based
on the PDH method, but use a slightly different approach.  Both the
existing and proposed systems lock to a vapor cell.  The
existing locking unit uses an acousto-optic modulator, which uses acoustics
to produce a frequency-shifted diffraction pattern.  This pattern is filtered
through a vapor cell, and then coupled into a photosensor.  From there,
frequency-shifted parts are extracted, which gives us an error signal.  This
error signal locks the diode laser's control system to the resonant frequency
of the chosen element. \\

Acousto-optic modulators in this application are effectively limited to a
modulation frequency of about 200 kHz. After filtering and processing, this
results in a system response speed  on the order of 10kHz. This response speed
has a direct impact on the resultant linewidth of the laser.  \\

Electro-optic modulators explot a different effect, the Pockels Effect, and,
for all intents and purposes, do not do not experience this limitation. A
locking unit based on the PH method and using an electro-optic modulator will
be built, and its performance will be compared to the existing AOM-based
locking unit. There will be three stages to this project:
\begin{enumerate}
 \item Measuring the operating parameters of the existing system
 \item Building the electro-optic modulator based system
 \item Measuring the operating parameters of the new system, using the
 old system as a benchmark.
\end{enumerate}
This project is technically demanding, but the technical expertise of this
group, as well as close collaboration with the sponsor, should result in
completion.

